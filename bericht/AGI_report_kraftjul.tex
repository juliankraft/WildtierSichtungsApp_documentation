\documentclass{article}
\usepackage[margin=2.5cm]{geometry} % Set margins
\usepackage{graphicx}
\usepackage[absolute]{textpos} % Enable absolute positioning
\usepackage{titlesec} % Package for controlling section title appearance
\usepackage[scaled]{helvet}
\usepackage[T1]{fontenc}
\usepackage{fancyhdr}
\usepackage[utf8]{inputenc}
\usepackage{lmodern}
\usepackage{amsmath}

\usepackage{url} % Load the url package

% Set up hyperref
\usepackage{hyperref}
\hypersetup{
    colorlinks=true,
    linkcolor=blue,
    urlcolor=blue,
    linkbordercolor={0 0 1},
    linktoc=none
}

% Set up references
\usepackage[
    backend=biber,             % Use biber backend (an external tool)
    sorting=none,              % Enumerates the reference in order of their appearance
    style=authoryear           % Choose here your preferred citation style
]{biblatex}
\addbibresource{bibliography.bib} % The filename of the bibliography
\usepackage[autostyle=true]{csquotes} 
                               % Required to generate language-dependent quotes 
                               % in the bibliography

\usepackage[german]{babel}     % Set the document language to German                               

\setlength{\TPHorizModule}{1cm} % Set horizontal unit of measure
\setlength{\TPVertModule}{1cm} % Set vertical unit of measure
\setlength{\parindent}{0pt}

\renewcommand{\familydefault}{\sfdefault}

\makeatletter
\renewcommand{\maketitle}{
  \begin{flushleft} 
    \Large\textmd{\@title} 
    \par
  \end{flushleft}
}
\makeatother

% Define style of sectiontitles
\titleformat{\section}
  {\normalfont\Large\mdseries}{\thesection}{1em}{}

\titleformat{\subsection}
  {\normalfont\large\mdseries} % Adjust style: smaller size, medium series (not italic)
  {} % No label
  {0pt} % Spacing between label and title
  {} % Code to execute after the title
\titlespacing*{\subsection}
  {0pt} % Left margin
  {0.8em} % Space above
  {0.4em} % Space below

% Set up fancyhdr
\fancyhf{} % Clear all headers and footers
\renewcommand{\headrulewidth}{0pt} % Remove the header rule
\rfoot{\thepage} % Place the page number in the right footer
\pagestyle{fancy}

% Add listings package for code highlighting
\usepackage{listings}
\usepackage{xcolor}
\usepackage{tcolorbox}

%%%%% Title %%%%%
\title{Individuelle Applikation zur georeferenzierten Datenerfassung}

\begin{document}

%%%%% Header %%%%%
\begin{textblock}{1}(2.5,1) % Position 1cm from left and 1cm from top
        \includegraphics[width=6cm]{logo.jpg} % Add logo
\end{textblock}

\begin{textblock}{6}(13,1) % Position 14cm from left and 1cm from top
        \raggedleft
        Annalisa Berger,
        Jonas Iseli, \\
        Pascal Kipf,
        Julian Kraft\\
        Digitale Agrodiagnostik\\
        \today
\end{textblock}

\vspace*{1.5cm}

%%%%% Document %%%%%

\maketitle

\tableofcontents


\section{Idee} %%%%%

Das Erfassen von georeferenzierten Daten ist eine häufige Aufgabe bei Forschung im Bereich 
Umweltnaturwissenschaften. Leider werden dabei noch oft Papier und Stift verwendet, was
die spätere Auswertung und Verarbeitung der Daten erschwert. Die Idee ist es, einen Bausatz für Applikationen
zu entwickeln, der es ermöglicht, massgeschneiderte Applikationen für die georeferenzierte Datenerfassung
zu erstellen. Die Applikation soll es ermöglichen, georeferenzierte Daten direkt auf dem Feld mit dem Smartphone
zu erfassen. Die Georeferenzierung soll dabei automatisch mit Hilfe des GPS-Moduls des Smartphones
erfolgen. Die Daten sollen in einer eigenen Datenbank gespeichert werden, um die Datensicherheit zu gewährleisten.

\section{Vorprojekt} %%%%%

\subsection{Marktanalyse} %----

Eine Marktanalyse hat ergeben, dass es bereits einige Applikationen gibt, die ähnliche Funktionen
anbieten. Beispiele dafür sind: \href{https://www.fastfieldforms.com/}{Fastfield},
\href{https://merginmaps.com/}{Mergin Maps}, \href{https://www.fulcrumapp.com/}{Fulcrum} und 
diverse mehr. Diese Applikationen sind jedoch meistens sehr teuer und bieten viele Funktionen, die
für den Anwendungszweck nicht benötigt werden. Dazu kommt im Zusammenhang mit Forschung die
Problematik der Datensicherheit. In fast allen Fällen werden die Daten auf dem Server des Anbieters
gespeichert, was für Forschungszwecke nicht immer ideal ist.

\subsection{Nutzen für die Anwender} %----

Das Erfassen von georeferenzierten Daten muss nicht mehr mühsam mit Papier und Stift erfolgen - somit
fällt das spätere Übertragen der Daten in eine digitale Form weg. Dadurch wird die Fehleranfälligkeit
der Datenerfassung reduziert. Die Daten sind sicher in einer eigenen Datenbank gespeichert und die
Kontrolle über die Daten bleibt beim Anwender.

\subsection{Marketingstrategie} %----

Die Idee ist es, den Bausatz für die Applikation open-source zu veröffentlichen. Jeder, der in der Lage ist,
damit selbst eine Applikation zu erstellen, kann dies tun. Für Anwender, die nicht in der Lage sind,
damit eine für sie massgeschneiderte Applikation zu erstellen, wird ein Service angeboten, bei dem
die Applikation für sie erstellt wird. Dieser Service wird kostenpflichtig sein und wird nach Aufwand
verrechnet. Wann immer möglich, soll der Service aber mit den IT Ressourcen der Anwender beziehungsweise
der Forschungseinrichtung durchgeführt werden. Da das ganze nicht kommerziell ist, wird keine Werbung
geschaltet. Lediglich sollen potentielle Anwender über die Existenz des Bausatzes und die Möglichkeit
für Unterstützung informiert werden.

\subsection{Technische Machbarkeit} %----

Die technische Machbarkeit wurde bereits mittels eines Prototyps überprüft, der eine einfache Applikation
zur Erfassung von Wildtiersichtungen ermöglicht. Die Technische Machbarkeit auf der Seite der
Lokalen Anwendung auf dem Smartphone wurde bestätigt. Die Machbarkeit, das ganze in die IT Infrastruktur
einer Forschungseinrichtung zu integrieren, muss von Fall zu Fall mit den entsprechenden IT-Verantwortlichen
geklärt werden.

\section{Projekt} %%%%%

\subsection{Name und Branding} %----

Da es sich nicht um ein kommerzielles Projekt handelt, wird auf ein Branding verzichtet. Die Funktionalität
und die Einfachheit der Applikation sollen im Vordergrund stehen. Es soll keinerlei Ablenkungen vom
wesentlichen Zweck der Applikation geben. Ein Name für die Applikation kann natürlich bei jeder Anwendung individuell gewählt werden. 
Für diesen Prototyp wurde der äusserst kreative Name "Wildtierapp" gewählt.\\

Etwas mehr Kreativität hat die Designabteilung unter der Leitung von Annalisa Berger bei der Gestaltung des Logos
an den Tag gelegt. Das fertige Logo für die Applikation sieht wie folgt aus: \\

\begin{center}
  \includegraphics[width=0.3\textwidth]{"logo_app.png"}
\end{center}

Verschiedene Elemente des wesentlichen Zwecks der Applikation sind im Logo enthalten. Es wird ein Geo-Tag und die Erde als Karten Grid dargestellt
um den Zweck der georeferenzierten Datenerfassung zu symbolisieren. Der Hirsch und der Adler sind Symbole für die Wildtiere,
die den Inhalt der Daten repräsentieren. Die Idee ist es, dass für jede Anwendung eine Ableitung
dieses Logos verwendet werden kann, indem der Hirsch und der Adler durch Symbole ausgetauscht werden, die
für die spezifische Anwendung relevant sind.

\newpage

\subsection{Prototyp} %----

Der Prototyp ist eine einfache Applikation zur Erfassung von Wildtiersichtungen. Die Applikation
wird auf einem privaten Server gehostet und die Daten werden in einer Datenbank auf demselben Server gespeichert.\\

Die Applikation kann unter folgendem Link aufgerufen werden - allerdings funktioniert es nur mit dem Google Chrome Browser:\\

\begin{center}

  \includegraphics[width=3cm]{qr_code_app.png}\\
  \url{https://wildtierapp.juliankraft.ch/app/}

\end{center}

Die erhobenen Daten können unter folgendem Link eingesehen werden:\\

\begin{center}

  \includegraphics[width=3cm]{qr_code_data.png}\\
  \url{https://wildtierapp.juliankraft.ch/}

\end{center}

Für den effektiven Datenzugriff um diese weiter verarbeiten zu können gibt es einen API Endpunkt um auf die SQL Datenbank zuzugreifen.


\newpage

\subsection{SWOT-Analyse} %----

\begin{minipage}{0.5\textwidth}
  \begin{tcolorbox}[title=Strengths (Stärken), colframe=blue!50!black, colback=blue!10, height=6cm]
    \begin{itemize}
        \item Sehr individuell
        \item einfach und beschränkt auf das Wesentliche
        \item Absolute Kontrolle über die Daten
        \item Open-Source
    \end{itemize}
  \end{tcolorbox}
\end{minipage}
\hfill
\begin{minipage}{0.5\textwidth}
  \begin{tcolorbox}[title=Weaknesses (Schwächen), colframe=red!50!black, colback=red!10, height=6cm]
    \begin{itemize}
        \item Browser basiert
        \item Programmierkenntnisse notwendig
        \item limitierte Funktionalität
        \item Läuft aktuell noch mit dem Kartenservice von Google (Google Maps)
    \end{itemize}
  \end{tcolorbox}
\end{minipage}

\vspace{1cm}

\begin{minipage}{0.5\textwidth}
  \begin{tcolorbox}[title=Opportunities (Chancen), colframe=green!50!black, colback=green!10, height=6cm]
    \begin{itemize}
        \item Open-Source - andere könnten Funktionalität erweitern und verbessern
        \item Unabhängigkeit von kommerziellen Anbietern
        \item Möglichkeit für Forschungseinrichtungen, eigene Applikationen zu erstellen
    \end{itemize}
  \end{tcolorbox}
\end{minipage}
\hfill
\begin{minipage}{0.5\textwidth}
  \begin{tcolorbox}[title=Threats (Gefahren), colframe=orange!50!black, colback=orange!10, height=6cm]
    \begin{itemize}
        \item Niemand interessiert sich dafür
        \item es könnte abschreckend wirken, dass Programmierkenntnisse notwendig sind
        \item Konkurrenz durch kommerzielle Anbieter
    \end{itemize}
  \end{tcolorbox}
\end{minipage}

\newpage

\section{Umsetzung} %%%%%

\subsection{Zeitplan} %----

Da schon ein Prototyp existiert, kann die Umsetzung relativ schnell erfolgen. Als nächster Schritt wird
ein konkretes Projekt benötigt, wo tatsächlich georeferenzierte Daten erfasst werden müssen. Die Applikation
wird dann entsprechend angepasst und erweitert. Die Umsetzung wird in enger Zusammenarbeit mit den
Anwendern erfolgen, um sicherzustellen, dass die Applikation den Anforderungen entspricht.
Falls ein Projekt gefunden wird, und das Hosting auf einem Server der Forschungseinrichtung möglich ist,
oder einer anderen geeigneten Lösung gefunden wurde, kann die Umsetzung in einem Monaten abgeschlossen sein.

\subsection{Budget} %----

Für die Umsetzung eines konkreten Projekts wird der Lohn für einen Programmierer für ca. einen Monat benötigt.
Dazu kommen noch allfällige Kosten für das Hosting der Applikation. Die Kosten dafür sind gering und
abhängig von der Laufzeit des Projekts - Server, die das handeln können sind schon für rund 50.-/Monat zu haben.
Je nach Grösse des Projekts können auch noch Kosten für den
Kartenservice von Google anfallen - dies kann aber auch umgegangen werden, indem ein anderer Kartenservice
wie zum Beispiel OpenStreetMap verwendet wird.

\subsection{Ausblick} %----

Da die Applikation open-source ist, besteht die Möglichkeit, dass andere Forschungseinrichtungen
sie ebenfalls nutzen. Dabei ist zu hoffen, dass die Funktionalität der Applikation durch die
Beiträge anderer verbessert wird. Es könnten auch neue Funktionalitäten hinzugefügt werden wie zum
Beispiel die Möglichkeit, Bilder zu den Daten hinzuzufügen.

\section*{Deklaration zu AI Tool Nutzung} %%%%%

Zum Erstellen dieses Dokumentes wurde ChatGPT und GitHub copilot verwendet.
Auch bei der Programmierung des Prototyps kam vie AI zum Einsatz.

\end{document}
