\documentclass{article}
\usepackage[margin=2.5cm]{geometry} % Set margins
\usepackage{graphicx}
\usepackage[absolute]{textpos} % Enable absolute positioning
\usepackage{titlesec} % Package for controlling section title appearance
\usepackage[scaled]{helvet}
\usepackage[T1]{fontenc}
\usepackage{fancyhdr}
\usepackage[utf8]{inputenc}
\usepackage{lmodern}
\usepackage{amsmath}

\usepackage{url} % Load the url package

% Set up hyperref
\usepackage{hyperref}
\hypersetup{
    colorlinks=true,
    linkcolor=blue,
    urlcolor=blue,
    linkbordercolor={0 0 1},
    linktoc=none
}

% Set up references
\usepackage[
    backend=biber,             % Use biber backend (an external tool)
    sorting=none,              % Enumerates the reference in order of their appearance
    style=numeric           % Choose here your preferred citation style
]{biblatex}
\addbibresource{agi_projekt.bib} % The filename of the bibliography

\usepackage[autostyle=true]{csquotes} 
                               % Required to generate language-dependent quotes 
                               % in the bibliography
                               
% Customize citation number color
\usepackage{xcolor} % Load xcolor for color customization

% Change the color of citation numbers and brackets in text
\renewcommand*{\mkbibbrackets}[1]{\textcolor{blue}{[#1]}} % Brackets and numbers in blue

% Change the color of numbers in the bibliography
\DeclareFieldFormat{labelnumber}{\textcolor{blue}{#1}}


\usepackage[german]{babel}     % Set the document language to German                               

\setlength{\TPHorizModule}{1cm} % Set horizontal unit of measure
\setlength{\TPVertModule}{1cm} % Set vertical unit of measure
\setlength{\parindent}{0pt}

\renewcommand{\familydefault}{\sfdefault}

\makeatletter
\renewcommand{\maketitle}{
  \begin{flushleft} 
    \Large\textmd{\@title} 
    \par
  \end{flushleft}
}
\makeatother

% Define style of sectiontitles
\titleformat{\section}
  {\normalfont\Large\mdseries}{\thesection}{1em}{}

\titleformat{\subsection}
  {\normalfont\large\mdseries} % Adjust style: smaller size, medium series (not italic)
  {} % No label
  {0pt} % Spacing between label and title
  {} % Code to execute after the title
\titlespacing*{\subsection}
  {0pt} % Left margin
  {0.8em} % Space above
  {0.4em} % Space below

% Set up fancyhdr
\fancyhf{} % Clear all headers and footers
\renewcommand{\headrulewidth}{0pt} % Remove the header rule
\rfoot{\thepage} % Place the page number in the right footer
\pagestyle{fancy}

% Add listings package for code highlighting
\usepackage{listings}
\usepackage{xcolor}
\usepackage{tcolorbox}

%%%%% Title %%%%%
\title{WebApp zur Erfassung von Wildtiersichtungen}

\begin{document}

%%%%% Header %%%%%
\begin{textblock}{1}(2.5,1) % Position 1cm from left and 1cm from top
        \includegraphics[width=6cm]{images/logo.jpg} % Add logo
\end{textblock}

\begin{textblock}{6}(13,1) % Position 14cm from left and 1cm from top
        \raggedleft
        Julian Kraft UI22\\
        Angewandte Geoinformatik\\
        \today
\end{textblock}

\vspace*{1.5cm}

%%%%% Document %%%%%

\maketitle

\tableofcontents


\section{Einleitung} %%%%%

Im Zentrum jedes Projekts mit Geodaten stehen die Daten. Wenn man diese noch nicht hat, geht es darum eine
geeignete Methode zu finden, um diese zu erfassen. Im Bereich Umweltnaturwissenschaften sind das noch
immer oft manuell erhobene Daten. Auch diese Feldarbeit kann jedoch durch den Einsatz von Technologie
vereinfacht werden. Ein geeignetes Tool dafür besitzt so ziemlich jeder - das Smartphone.
Eine kurze Online-Recherche zeigt, dass es bereits viele Apps gibt, die das können. Jedoch sind diese
oft relativ teuer, können viel mehr als das was man braucht und die Datenhoheit ist oft beim Anbieter der App.
Bei all den Modernen Frameworks und Programmiersprachen kann das doch nicht so schwer sein, eine eigene App
zu schreiben, die genau das macht, was man will. Und so entstand die Idee, eine eigene App zu erstellen.
In dieser Arbeit soll geprüft werden, wie schwer es wirklich ist, eine eigene App zu erstellen und das
mit sehr begrenzten Vorkenntnissen, viel Trial and Error, unglaublich viel Recherchen, dem
Einsatz von AI, viel Geduld und etwas bis etwas mehr Unterstützung von einem erfahrenen Entwickler.
Genau mit dieser Unterstützung wurde dann auch begonnen. Um nicht alles von Grund auf herausfinden zu müssen,
wurde mit einem Beratungsgespräch begonnen. Dabei wurde die Idee vorgestellt. Es wurde besprochen, welche
Technologien, Programmiersprachen und Frameworks zielführend, machbar und sinnvoll sind.
In einem weiteren Termin wurde dann gemeinsam der Server konfiguriert und die Entwicklungsumgebung eingerichtet.
Ein nicht so kleiner Crashkurs hat dann einen guten Einstieg ins Projekt erleichtert.

\section{Methoden} %%%%%

Die App besteht im Wesentlichen aus drei Teilen: Dem webbasierten Frontend, dem Backend Server und der Datenbank.
Diese drei Teile sind miteinander verbunden und kommunizieren miteinander. Das Frontend ist die Benutzeroberfläche,
die man auf dem Smartphone zu sehen bekommt. Das Backend ist die Logik, die im Hintergrund abläuft und die Datenbank
ist der Speicherort für alle Daten, die abgerufen oder gespeichert werden.

\subsection{Frontend} %%%%%

Das Frontend wurde mit dem Framework Ionic erstellt. Ionic ist ein open source toolkit für die Entwicklung von
mobile User Interfaces. Es basiert auf Angular, einem weiteren open source Framework, das von Google entwickelt wurde.
Es ermöglicht die Erstellung von Webanwendungen, die auf allen Plattformen - Android, iOS und Web - laufen und dabei
eine native App-ähnliche Performance und Erfahrung bieten. \autocite{IonicFrameworkCrossPlatform}
Das Framework ermöglicht es relativ einfach ein Projekt zu starten und in einer Entwicklungsumgebung zu testen.
Dabei wird die App im Browser angezeigt. Das effektive programmieren geschieht dann in den Programmiersprachen
HTML, CSS und TypeScript. TypeScript ist eine von Microsoft entwickelte Programmiersprache, die auf JavaScript basiert.
CSS ist eine Stylesheet-Sprache, die für das Design der Benutzeroberfläche verwendet wird. HTML ist eine Markup-Sprache,
die für die Struktur der Benutzeroberfläche verwendet wird.

\subsection{Backend} %%%%%

Das Backend wurde mit der Programmiersprache Go erstellt. Go ist eine von Google entwickelte Programmiersprache,
die für die Entwicklung von Webanwendungen und Servern verwendet wird. Sie ist einfach zu lernen und zu verwenden
und bietet eine gute Performance \autocite{GoProgrammingLanguage}. Auch soll erwähnt sein, dass GitHub Copilot
für die Programmiersprache Go ausgezeichnete Vorschläge macht und so das Programmieren massiv erleichtert.
Die Wesentlichen Logiken, die vom Backend in der App ablaufen sind die Authentifizierung des Benutzers, den Zugriff
auf die Datenbank um beispielsweise verfügbare User zu prüfen oder die Auswahlmöglichkeiten bei den Tieren abzurufen
und das Transformieren und Speichern von neuen Daten in der Datenbank.

\subsection{Datenbank} %%%%%

Die verwendete Datenbank ist MariaDB. MariaDB ist eine von MySQL abgeleitete relationale Datenbank, die von der
Open-Source-Community entwickelt wird. Sie ist einfach zu installieren und zu verwenden und bietet eine gute Performance \autocite{MariaDBFoundation}.
In diesem Fall wurde Docker verwendet, um die Datenbank auf dem Server zu installieren und in einem 
sogenannten Container zu betreiben. Die Datenbank besteht aus drei Tabellen: User, Animals und Sightings.

\section{Resultate} %%%%%

\section{Diskussion} %%%%%


\section{Acknowledgements und Deklaration} %%%%%

Danke an Ramon Ott für die Unterstützung bei der Umsetzung dieses Projektes.

Danke an Annalisa Berger für das Design des Logos.

Zum Erstellen dieses Dokumentes wurde ChatGPT und GitHub copilot verwendet.
Auch bei der Programmierung des Prototyps kam vie AI zum Einsatz.

\printbibliography

\end{document}
